% vim: set tabstop=8:softtabstop=4:shiftwidth=4:textwidth:78:formatoptions+=t% ===> this file was generated automatically by noweave --- better not edit it
\section{LDPC Decoder Core Top-level}
The LDPC core integrates the bit and check nodes as per the topology specified
by the $H$ matrix.

\nwfilename{LdpcCore.bsv.nw}\nwbegincode{1}\sublabel{NW3fuPVj-To6Q8-1}\nwmargintag{{\nwtagstyle{}\subpageref{NW3fuPVj-To6Q8-1}}}\moddef{boilerplate~{\nwtagstyle{}\subpageref{NW3fuPVj-To6Q8-1}}}\endmoddef\nwstartdeflinemarkup\nwenddeflinemarkup
// EE-705 Course Project -- LDPC Decoder

package LdpcCore;

// -----------------------------------------------------------------
// This package defines:
//
//    LdpcCore    : Interface of the LDPC Core module
//    mkLdpcCore  : Top-level module of the LDPC core
//
//    v1.0        : Non-pipelined. The nodes can handle one code-word
//                  at a time
//
// -----------------------------------------------------------------

import GetPut           :: *;
import ClientServer     :: *;
import FIFO             :: *;
import Vector           :: *;
import LdpcTypes        :: *;
import Nodes            :: *;

\LA{}interface definition~{\nwtagstyle{}\subpageref{NW3fuPVj-3l1BOL-1}}\RA{}

\LA{}LdpcCore module definition~{\nwtagstyle{}\subpageref{NW3fuPVj-lFjoa-1}}\RA{}

\LA{}conn module definition~{\nwtagstyle{}\subpageref{NW3fuPVj-2insK6-1}}\RA{}

\nwnotused{boilerplate}\nwendcode{}\nwbegindocs{2}\nwdocspar
Global types for the LDPC decoder are defined in {\Tt{}LdpcTypes\nwendquote} package. The
following types are global:

\begin{description}
\item[Symbol:] The bit pattern describing a soft-value
\item[DataWord:] A collection of soft-value forming the code word or the
decoded word
\item[CodeLength:] The number of symbols in a DataWord. In this configuration,
this is equal to the number of bit nodes and check nodes
\item[NBitNodes:] The number of bit nodes. In this design each bit
node manipulates a symbol. So, the number of bit nodes is same as the
number of symbols.
\item[NCheckNodes:] The number of check nodes. In this design each
check node manipulates a symbol. So, the number of check nodes is same as the
number of symbols.
\item[NConnections:] Number of 1s in a row of the H matrix. Represents the
fanouts/fanin from/to a check-node. Is also the number of 1s in a column of the
H matrix, which represents the fanouts/fanin from/to a bit node.
\item[NIteration:] Number of iterations between bit node and check
nodes.
\end{description}

\nwenddocs{}\nwbegincode{3}\sublabel{NW3fuPVj-4O0RoX-1}\nwmargintag{{\nwtagstyle{}\subpageref{NW3fuPVj-4O0RoX-1}}}\moddef{type definitions~{\nwtagstyle{}\subpageref{NW3fuPVj-4O0RoX-1}}}\endmoddef\nwstartdeflinemarkup\nwenddeflinemarkup
package LdpcTypes;
import Vector :: *;

// Global types used across the design
// Soft-value bit pattern
typedef Bit #(4) Symbol;

// Number of symbols in a codeword
typedef 7 CodeLength;

// A coded/decoded word
typedef Vector #(CodeLength, Symbol) DataWord;

// Number of bit nodes
typedef CodeLength NBitNodes;

// Number of check nodes
typedef CodeLength NCheckNodes;

// Fanouts from an individual check-node. This is the number of 1s in a row of
// the H-matrix
typedef 3 NConnections;

// Number of iterations between bit nodes and check nodes
typedef 8 NIterations;
endpackage : LdpcTypes

\nwnotused{type definitions}\nwendcode{}\nwbegindocs{4}\nwdocspar
The LDPC decoder top-level provides a Server interface which receives a
code word as request and responds with the decoded code word.

\nwenddocs{}\nwbegincode{5}\sublabel{NW3fuPVj-3l1BOL-1}\nwmargintag{{\nwtagstyle{}\subpageref{NW3fuPVj-3l1BOL-1}}}\moddef{interface definition~{\nwtagstyle{}\subpageref{NW3fuPVj-3l1BOL-1}}}\endmoddef\nwstartdeflinemarkup\nwusesondefline{\\{NW3fuPVj-To6Q8-1}}\nwenddeflinemarkup
typedef Server #(DataWord, DataWord) LdpcCore;

// -----------------------------------------------------------------

\nwused{\\{NW3fuPVj-To6Q8-1}}\nwendcode{}\nwbegindocs{6}\nwdocspar
\section{The LDPC Core Top-level}
The number of bit-nodes and check-nodes depend on the topology of the
tanner graph. For a given topology, the number of bit-nodes and the
number of check-nodes are inputs. These are defined in {\Tt{}LdpcTypes\nwendquote}.

\nwenddocs{}\nwbegincode{7}\sublabel{NW3fuPVj-lFjoa-1}\nwmargintag{{\nwtagstyle{}\subpageref{NW3fuPVj-lFjoa-1}}}\moddef{LdpcCore module definition~{\nwtagstyle{}\subpageref{NW3fuPVj-lFjoa-1}}}\endmoddef\nwstartdeflinemarkup\nwusesondefline{\\{NW3fuPVj-To6Q8-1}}\nwenddeflinemarkup
(* synthesize *)
module mkLdpcCore (LdpcCore);
   \LA{}state~{\nwtagstyle{}\subpageref{NW3fuPVj-3wDyIc-1}}\RA{}
   \LA{}rules~{\nwtagstyle{}\subpageref{NW3fuPVj-1igLiJ-1}}\RA{}
   \LA{}interfaces~{\nwtagstyle{}\subpageref{NW3fuPVj-2uCSY5-1}}\RA{}
   \LA{}wrap up mkLdpcCore~{\nwtagstyle{}\subpageref{NW3fuPVj-1YmVkX-1}}\RA{}

\nwused{\\{NW3fuPVj-To6Q8-1}}\nwendcode{}\nwbegindocs{8}\nwdocspar
The bit-nodes and check-nodes are instantiated as vectors of modules
with the number of elements defined by the numeric types
{\Tt{}NBitNodes\nwendquote} and {\Tt{}NCheckNodes\nwendquote}.

\nwenddocs{}\nwbegincode{9}\sublabel{NW3fuPVj-3wDyIc-1}\nwmargintag{{\nwtagstyle{}\subpageref{NW3fuPVj-3wDyIc-1}}}\moddef{state~{\nwtagstyle{}\subpageref{NW3fuPVj-3wDyIc-1}}}\endmoddef\nwstartdeflinemarkup\nwusesondefline{\\{NW3fuPVj-lFjoa-1}}\nwprevnextdefs{\relax}{NW3fuPVj-3wDyIc-2}\nwenddeflinemarkup
// Bit nodes
Vector #(NBitNodes, BitNode) vBitNodes <- replicateM (mkBitNode);

// Check nodes
Vector #(NCheckNodes, CheckNode) vCheckNodes <- replicateM (mkCheckNode);

\nwalsodefined{\\{NW3fuPVj-3wDyIc-2}}\nwused{\\{NW3fuPVj-lFjoa-1}}\nwendcode{}\nwbegindocs{10}\nwdocspar
Finally, we will need a pair of FIFOs to accept the code word on the
input side and buffer the decoded word on the output side.

\nwenddocs{}\nwbegincode{11}\sublabel{NW3fuPVj-3wDyIc-2}\nwmargintag{{\nwtagstyle{}\subpageref{NW3fuPVj-3wDyIc-2}}}\moddef{state~{\nwtagstyle{}\subpageref{NW3fuPVj-3wDyIc-1}}}\plusendmoddef\nwstartdeflinemarkup\nwusesondefline{\\{NW3fuPVj-lFjoa-1}}\nwprevnextdefs{NW3fuPVj-3wDyIc-1}{\relax}\nwenddeflinemarkup
// Input and output FIFOs
FIFO #(DataWord) ffI <- mkFIFO;
FIFO #(DataWord) ffO <- mkFIFO;

\nwused{\\{NW3fuPVj-lFjoa-1}}\nwendcode{}\nwbegindocs{12}\nwdocspar
At the top-level every rule reads from the transmitting end (bit or
check node), and writes to the receiving end (bit or check node). These
{\Tt{}connections\nwendquote} may also be autogenerated from a high-level script. The
following connections are as per the matrix:

\[
H = \begin{bmatrix}
1 1 0 1 0 0 0 \\
0 1 1 0 1 0 0 \\
0 0 1 1 0 1 0 \\
0 0 0 1 1 0 1 \\
1 0 0 0 1 1 0 \\
0 1 0 0 0 1 1 \\
1 0 1 0 0 0 1 \\
\end{bmatrix}
\]

\nwenddocs{}\nwbegincode{13}\sublabel{NW3fuPVj-1igLiJ-1}\nwmargintag{{\nwtagstyle{}\subpageref{NW3fuPVj-1igLiJ-1}}}\moddef{rules~{\nwtagstyle{}\subpageref{NW3fuPVj-1igLiJ-1}}}\endmoddef\nwstartdeflinemarkup\nwusesondefline{\\{NW3fuPVj-lFjoa-1}}\nwprevnextdefs{\relax}{NW3fuPVj-1igLiJ-2}\nwenddeflinemarkup
// connect up all the bit nodes and check nodes
// bit-node to check-node connections
mkConnMulti (
     vBitNodes[0].b2c
   , vCheckNodes[0].b2c[0], vCheckNodes[4].b2c[0], vCheckNodes[6].b2c[0]);
mkConnMulti (
     vBitNodes[1].b2c
   , vCheckNodes[0].b2c[1], vCheckNodes[1].b2c[0], vCheckNodes[5].b2c[0]);
mkConnMulti (
     vBitNodes[2].b2c
   , vCheckNodes[1].b2c[1], vCheckNodes[2].b2c[0], vCheckNodes[6].b2c[1]);
mkConnMulti (
     vBitNodes[3].b2c
   , vCheckNodes[0].b2c[2], vCheckNodes[2].b2c[1], vCheckNodes[3].b2c[0]);
mkConnMulti (
     vBitNodes[4].b2c
   , vCheckNodes[1].b2c[2], vCheckNodes[3].b2c[1], vCheckNodes[4].b2c[1]);
mkConnMulti (
     vBitNodes[5].b2c
   , vCheckNodes[2].b2c[2], vCheckNodes[4].b2c[2], vCheckNodes[5].b2c[1]);
mkConnMulti (
     vBitNodes[6].b2c
   , vCheckNodes[3].b2c[2], vCheckNodes[5].b2c[2], vCheckNodes[6].b2c[2]);

// check-node to bit-node connections
mkConnMulti (
     vCheckNodes[0].c2b
   , vBitNodes[0].c2b[0], vBitNodes[1].c2b[0], vBitNodes[3].c2b[0]);

mkConnMulti (
     vCheckNodes[1].c2b
   , vBitNodes[1].c2b[1], vBitNodes[2].c2b[0], vBitNodes[4].c2b[0]);

mkConnMulti (
     vCheckNodes[2].c2b
   , vBitNodes[2].c2b[1], vBitNodes[3].c2b[1], vBitNodes[5].c2b[0]);

mkConnMulti (
     vCheckNodes[3].c2b
   , vBitNodes[3].c2b[2], vBitNodes[4].c2b[1], vBitNodes[6].c2b[0]);

mkConnMulti (
     vCheckNodes[4].c2b
   , vBitNodes[0].c2b[1], vBitNodes[4].c2b[2], vBitNodes[5].c2b[1]);

mkConnMulti (
     vCheckNodes[5].c2b
   , vBitNodes[1].c2b[2], vBitNodes[5].c2b[2], vBitNodes[6].c2b[1]);

mkConnMulti (
     vCheckNodes[6].c2b
   , vBitNodes[0].c2b[2], vBitNodes[2].c2b[2], vBitNodes[6].c2b[2]);

\nwalsodefined{\\{NW3fuPVj-1igLiJ-2}\\{NW3fuPVj-1igLiJ-3}}\nwused{\\{NW3fuPVj-lFjoa-1}}\nwendcode{}\nwbegindocs{14}\nwdocspar
The connections between bit and check nodes take care of all the
communication between the nodes for each iteration. These connections have
been abstracted in the module {\Tt{}mkConnMulti\nwendquote}. This module
takes the interfaces being connected as arguments and returns the logic to
connect them up.

\nwenddocs{}\nwbegincode{15}\sublabel{NW3fuPVj-2insK6-1}\nwmargintag{{\nwtagstyle{}\subpageref{NW3fuPVj-2insK6-1}}}\moddef{conn module definition~{\nwtagstyle{}\subpageref{NW3fuPVj-2insK6-1}}}\endmoddef\nwstartdeflinemarkup\nwusesondefline{\\{NW3fuPVj-To6Q8-1}}\nwenddeflinemarkup
module mkConnMulti #(
     Get #(Symbol) src
   , Put #(Symbol) dst1
   , Put #(Symbol) dst2
   , Put #(Symbol) dst3) (Empty);

   \LA{}conn rule~{\nwtagstyle{}\subpageref{NW3fuPVj-8rWsf-1}}\RA{}

\LA{}conn wrap up~{\nwtagstyle{}\subpageref{NW3fuPVj-gadGr-1}}\RA{}

\nwused{\\{NW3fuPVj-To6Q8-1}}\nwendcode{}\nwbegindocs{16}\nwdocspar
One rule describes the connectable's behaviour. It connects
the outgoing symbol from the source node as incoming symbols to multiple
destination nodes.

\nwenddocs{}\nwbegincode{17}\sublabel{NW3fuPVj-8rWsf-1}\nwmargintag{{\nwtagstyle{}\subpageref{NW3fuPVj-8rWsf-1}}}\moddef{conn rule~{\nwtagstyle{}\subpageref{NW3fuPVj-8rWsf-1}}}\endmoddef\nwstartdeflinemarkup\nwusesondefline{\\{NW3fuPVj-2insK6-1}}\nwenddeflinemarkup
   // Connect a "SRC" to a "DST"
   rule rlConnect;
      let s <- src.get();
      dst1.put (s); dst2.put (s); dst3.put (s);
   endrule

\nwused{\\{NW3fuPVj-2insK6-1}}\nwendcode{}\nwbegindocs{18}\nwdocspar
In order to get things started, the input code-word needs to be input
to the bit nodes. The input code word will be sitting in the {\Tt{}ffI\nwendquote} FIFO.
In the current architecture, we are sending one symbol per bit-node. In
general, it is possible to fold multiple symbols onto a bit node but this
will require more complex nodes.

A code-word can be sent to all bit-nodes only when all of them are ready to
receive the code words.

\nwenddocs{}\nwbegincode{19}\sublabel{NW3fuPVj-1igLiJ-2}\nwmargintag{{\nwtagstyle{}\subpageref{NW3fuPVj-1igLiJ-2}}}\moddef{rules~{\nwtagstyle{}\subpageref{NW3fuPVj-1igLiJ-1}}}\plusendmoddef\nwstartdeflinemarkup\nwusesondefline{\\{NW3fuPVj-lFjoa-1}}\nwprevnextdefs{NW3fuPVj-1igLiJ-1}{NW3fuPVj-1igLiJ-3}\nwenddeflinemarkup
rule rlPutCodeWordIn;
   // Get the codeword from ffI
   let codeIn = ffI.first; ffI.deq;

   // Send each bit-node a symbol
   for (Integer i=0; i<valueOf(NBitNodes); i=i+1)
      vBitNodes[i].codeIn.put (codeIn[i]);
endrule

\nwused{\\{NW3fuPVj-lFjoa-1}}\nwendcode{}\nwbegindocs{20}\nwdocspar
At the output end, the decoded symbols are collected from the bit-nodes and
sent to the {\Tt{}ffO\nwendquote}. Again, one symbol each is connected from every
bit-node in this architecture. Decoded words can be collected from all the
bit-nodes only when all of them are ready with their decoded symbols.

\nwenddocs{}\nwbegincode{21}\sublabel{NW3fuPVj-1igLiJ-3}\nwmargintag{{\nwtagstyle{}\subpageref{NW3fuPVj-1igLiJ-3}}}\moddef{rules~{\nwtagstyle{}\subpageref{NW3fuPVj-1igLiJ-1}}}\plusendmoddef\nwstartdeflinemarkup\nwusesondefline{\\{NW3fuPVj-lFjoa-1}}\nwprevnextdefs{NW3fuPVj-1igLiJ-2}{\relax}\nwenddeflinemarkup
rule rlGetDecodedWordOut;
   DataWord dOut;

   // Get a symbol from each bit-node
   for (Integer i=0; i<valueOf(NBitNodes); i=i+1) begin
      let d <- vBitNodes[i].dataOut.get ();
      dOut [i] = d;
   end

   // Write the collected symbols into the output fifo
   ffO.enq (dOut);
endrule

\nwused{\\{NW3fuPVj-lFjoa-1}}\nwendcode{}\nwbegindocs{22}\nwdocspar
The {\Tt{}LdpcDecode\nwendquote} interface simply does the connections to the {\Tt{}ffI\nwendquote}
and {\Tt{}ffO\nwendquote} FIFOs.

\nwenddocs{}\nwbegincode{23}\sublabel{NW3fuPVj-2uCSY5-1}\nwmargintag{{\nwtagstyle{}\subpageref{NW3fuPVj-2uCSY5-1}}}\moddef{interfaces~{\nwtagstyle{}\subpageref{NW3fuPVj-2uCSY5-1}}}\endmoddef\nwstartdeflinemarkup\nwusesondefline{\\{NW3fuPVj-lFjoa-1}}\nwenddeflinemarkup
return (toGPServer (ffI, ffO));

\nwused{\\{NW3fuPVj-lFjoa-1}}\nwendcode{}\nwbegincode{24}\sublabel{NW3fuPVj-1YmVkX-1}\nwmargintag{{\nwtagstyle{}\subpageref{NW3fuPVj-1YmVkX-1}}}\moddef{wrap up mkLdpcCore~{\nwtagstyle{}\subpageref{NW3fuPVj-1YmVkX-1}}}\endmoddef\nwstartdeflinemarkup\nwusesondefline{\\{NW3fuPVj-lFjoa-1}}\nwenddeflinemarkup
endmodule : mkLdpcCore

\nwused{\\{NW3fuPVj-lFjoa-1}}\nwendcode{}

\nwixlogsorted{c}{{boilerplate}{NW3fuPVj-To6Q8-1}{\nwixd{NW3fuPVj-To6Q8-1}}}%
\nwixlogsorted{c}{{conn module definition}{NW3fuPVj-2insK6-1}{\nwixu{NW3fuPVj-To6Q8-1}\nwixd{NW3fuPVj-2insK6-1}}}%
\nwixlogsorted{c}{{conn rule}{NW3fuPVj-8rWsf-1}{\nwixu{NW3fuPVj-2insK6-1}\nwixd{NW3fuPVj-8rWsf-1}}}%
\nwixlogsorted{c}{{conn wrap up}{NW3fuPVj-gadGr-1}{\nwixu{NW3fuPVj-2insK6-1}\nwixd{NW3fuPVj-gadGr-1}}}%
\nwixlogsorted{c}{{interface definition}{NW3fuPVj-3l1BOL-1}{\nwixu{NW3fuPVj-To6Q8-1}\nwixd{NW3fuPVj-3l1BOL-1}}}%
\nwixlogsorted{c}{{interfaces}{NW3fuPVj-2uCSY5-1}{\nwixu{NW3fuPVj-lFjoa-1}\nwixd{NW3fuPVj-2uCSY5-1}}}%
\nwixlogsorted{c}{{LdpcCore module definition}{NW3fuPVj-lFjoa-1}{\nwixu{NW3fuPVj-To6Q8-1}\nwixd{NW3fuPVj-lFjoa-1}}}%
\nwixlogsorted{c}{{rules}{NW3fuPVj-1igLiJ-1}{\nwixu{NW3fuPVj-lFjoa-1}\nwixd{NW3fuPVj-1igLiJ-1}\nwixd{NW3fuPVj-1igLiJ-2}\nwixd{NW3fuPVj-1igLiJ-3}}}%
\nwixlogsorted{c}{{state}{NW3fuPVj-3wDyIc-1}{\nwixu{NW3fuPVj-lFjoa-1}\nwixd{NW3fuPVj-3wDyIc-1}\nwixd{NW3fuPVj-3wDyIc-2}}}%
\nwixlogsorted{c}{{type definitions}{NW3fuPVj-4O0RoX-1}{\nwixd{NW3fuPVj-4O0RoX-1}}}%
\nwixlogsorted{c}{{wrap up mkLdpcCore}{NW3fuPVj-1YmVkX-1}{\nwixu{NW3fuPVj-lFjoa-1}\nwixd{NW3fuPVj-1YmVkX-1}}}%
\nwbegincode{25}\sublabel{NW3fuPVj-gadGr-1}\nwmargintag{{\nwtagstyle{}\subpageref{NW3fuPVj-gadGr-1}}}\moddef{conn wrap up~{\nwtagstyle{}\subpageref{NW3fuPVj-gadGr-1}}}\endmoddef\nwstartdeflinemarkup\nwusesondefline{\\{NW3fuPVj-2insK6-1}}\nwenddeflinemarkup
endmodule
endpackage : LdpcCore
\nwused{\\{NW3fuPVj-2insK6-1}}\nwendcode{}
