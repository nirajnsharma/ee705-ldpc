\documentclass[11pt, a4paper]{article}
\usepackage{noweb}
\usepackage{amsmath}
\usepackage{a4wide}  % Set margins automatically for wider use of A4
\usepackage{graphicx}
\usepackage{placeins}   % to stop floats from floating
\usepackage{wrapfig}
\noweboptions{}
\title{EE705 Course Project -- LDPC Decoder}
\author{Anuja Singh (184074001), Niraj N Sharma (184077001), Sandeep Mishra (184076005) \and Sonali Shukla (184070014)}
\date{Interim Report: April, 2019}
\begin{document}
\maketitle
The project involves the design and layout of a LDPC decoder block. After
the creation of behavioural RTL we have chosen to use an open-source tool
based design flow to generate the GDS2 for the design.

\section{Block Diagram}
The LDPC decoder core represents the design which will be developed and
implemented upto GDS2. The LDPC decoder top-level represents the top-level
for functional verification on a Nano FPGA board using Quartus. 

\begin{figure}[hbt]
\includegraphics[width=\linewidth]{block-diagram.pdf}
\label{fig:bd}
\end{figure}

\section{Current Status and Plan Ahead}
The design of the behavioural verilog for the decoder is nearly complete. The
remaining steps are described below:
\begin{center}
   \begin{tabular}{lll}
      \hline
      Task & Tool & Status\\
      \hline
      Functional Verification & Quartus & Ongoing\\
      Synthesis to Gate level & Yosys & Ongoing\\
      CTS and STA & ?? & TBD\\
      Layout & Java Electric & TBD\\
      \hline
   \end{tabular}
\end{center}

The rest of the document describes the RTL design of the LDPC Core and its
components.

\include{LdpcTop}
% vim: set tabstop=8:softtabstop=4:shiftwidth=4:textwidth:78:formatoptions+=t% ===> this file was generated automatically by noweave --- better not edit it
\section{Nodes of the LDPC Decoder}
The work-horses of the decoder are the bit and check nodes. These iteratively
decode the input code-word passing the partially decoded result back-and-forth
for a fixed number of iterations.

%\author{Kim Jong Un}
%\begin{document}
%\maketitle

\nwfilename{Nodes.bsv.nw}\nwbegincode{1}\sublabel{NW1v8dKs-To6Q8-1}\nwmargintag{{\nwtagstyle{}\subpageref{NW1v8dKs-To6Q8-1}}}\moddef{boilerplate~{\nwtagstyle{}\subpageref{NW1v8dKs-To6Q8-1}}}\endmoddef\nwstartdeflinemarkup\nwenddeflinemarkup
// EE-705 Course Project -- LDPC Decoder

package Nodes;

// -----------------------------------------------------------------
// This package defines:
//
//    CheckNode   : Interface to the Check Nodes
//    BitNode     : Interface to the Bit Nodes
//    mkCheckNode : Micro-arch of the Check Node
//    mkBitNode   : Micro-arch of the Bit Node
//
//    v1.0        : The nodes can handle one code-word at a time
//
// -----------------------------------------------------------------

import ClientServer     :: *;
import GetPut           :: *;
import FIFO             :: *;
import Vector           :: *;
import LdpcTypes        :: *;

\LA{}type definitions~{\nwtagstyle{}\subpageref{NW1v8dKs-4O0RoX-1}}\RA{}

\LA{}module definition~{\nwtagstyle{}\subpageref{NW1v8dKs-1nQrPK-1}}\RA{}

\nwnotused{boilerplate}\nwendcode{}\nwbegindocs{2}\nwdocspar
The Bit-node and the Check-node are the two types of computation nodes in
the LDPC decoder. One iteration consists of processing at both of these
nodes. The interconnections between them is described by the incidence
matrix (H matrix).

\nwenddocs{}\nwbegincode{3}\sublabel{NW1v8dKs-4O0RoX-1}\nwmargintag{{\nwtagstyle{}\subpageref{NW1v8dKs-4O0RoX-1}}}\moddef{type definitions~{\nwtagstyle{}\subpageref{NW1v8dKs-4O0RoX-1}}}\endmoddef\nwstartdeflinemarkup\nwusesondefline{\\{NW1v8dKs-To6Q8-1}}\nwenddeflinemarkup
\LA{}interface definition~{\nwtagstyle{}\subpageref{NW1v8dKs-3l1BOL-1}}\RA{}

\nwused{\\{NW1v8dKs-To6Q8-1}}\nwendcode{}\nwbegindocs{4}\nwdocspar
\section{Interfaces of the Bit and Check Nodes}
Two interface types, one each for the Bit and Check nodes are defined in
this package. Both interfaces are parameterized by the numeric types
{\Tt{}NConnections\nwendquote}.

The numeric type, {\Tt{}NConnections\nwendquote} indicates the number of ones along a
particular row of the incidence matrix, which is also same as the number of
ones along a particular column of the incidence matrix.

\nwenddocs{}\nwbegincode{5}\sublabel{NW1v8dKs-3l1BOL-1}\nwmargintag{{\nwtagstyle{}\subpageref{NW1v8dKs-3l1BOL-1}}}\moddef{interface definition~{\nwtagstyle{}\subpageref{NW1v8dKs-3l1BOL-1}}}\endmoddef\nwstartdeflinemarkup\nwusesondefline{\\{NW1v8dKs-4O0RoX-1}}\nwprevnextdefs{\relax}{NW1v8dKs-3l1BOL-2}\nwenddeflinemarkup
// NConnections indicates the number of connections that a particular
// check-node has with the bit-nodes
interface CheckNode;
   interface Vector #(NConnections, Put #(Symbol)) b2c;
   interface Get #(Symbol) c2b;
endinterface

\nwalsodefined{\\{NW1v8dKs-3l1BOL-2}\\{NW1v8dKs-3l1BOL-3}}\nwused{\\{NW1v8dKs-4O0RoX-1}}\nwendcode{}\nwbegindocs{6}\nwdocspar
A check node can be imagined to receive {\Tt{}Symbol\nwendquote} values from a vector
of interfaces. Each interface is connected to a different bit nodes. In
turn, the check node returns a single {\Tt{}Symbol\nwendquote} values broadcast to all
its bit nodes.

\nwenddocs{}\nwbegincode{7}\sublabel{NW1v8dKs-3l1BOL-2}\nwmargintag{{\nwtagstyle{}\subpageref{NW1v8dKs-3l1BOL-2}}}\moddef{interface definition~{\nwtagstyle{}\subpageref{NW1v8dKs-3l1BOL-1}}}\plusendmoddef\nwstartdeflinemarkup\nwusesondefline{\\{NW1v8dKs-4O0RoX-1}}\nwprevnextdefs{NW1v8dKs-3l1BOL-1}{NW1v8dKs-3l1BOL-3}\nwenddeflinemarkup
// NConnections indicates the number of connections that a particular
// bit-node has with the check-nodes
interface BitNode;
   // Bit Node-Check Node Interface
   interface Vector #(NConnections, Put #(Symbol)) c2b;
   interface Get #(Symbol) b2c;

\nwused{\\{NW1v8dKs-4O0RoX-1}}\nwendcode{}\nwbegindocs{8}\nwdocspar
The inverse applies for the bit-check node connection from the bit
node's perspective. It sends a single {\Tt{}Symbol\nwendquote} values broadcast to all
its check nodes, and collects responses through a vector of interfaces,
each connected to a different check node.
 
\nwenddocs{}\nwbegincode{9}\sublabel{NW1v8dKs-3l1BOL-3}\nwmargintag{{\nwtagstyle{}\subpageref{NW1v8dKs-3l1BOL-3}}}\moddef{interface definition~{\nwtagstyle{}\subpageref{NW1v8dKs-3l1BOL-1}}}\plusendmoddef\nwstartdeflinemarkup\nwusesondefline{\\{NW1v8dKs-4O0RoX-1}}\nwprevnextdefs{NW1v8dKs-3l1BOL-2}{\relax}\nwenddeflinemarkup

   // External interfaces for receving code word and returning result
   interface Put     #(Symbol)   codeIn;
   interface Get     #(Symbol)   dataOut;

endinterface

// -----------------------------------------------------------------

\nwused{\\{NW1v8dKs-4O0RoX-1}}\nwendcode{}\nwbegindocs{10}\nwdocspar
In addition to the sub-interfaces to connect the check
and bit nodes, bit nodes also have the additional interfaces to receive the
code and send back the decoded data. Since each bit-node only deals with a
part of the code-word, it is sufficient to receive those symbols only. The
final decoded word is also constructed from the responses of the different
bit-nodes. Each bit node receives one symbol to decode at a time.

\nwenddocs{}\nwbegindocs{11}\nwdocspar
\section{The Bit Node}
The {\Tt{}mkBitNode\nwendquote} module receives the code word and initiates the
iterations.  Based on {\Tt{}NConnections\nwendquote} it is connected to a set of
{\Tt{}mkCheckNodes\nwendquote} representing the edges of the bipartite tanner graph. The
{\Tt{}mkBitNode\nwendquote} provides an interface of type {\Tt{}BitNode\nwendquote}.

\nwenddocs{}\nwbegincode{12}\sublabel{NW1v8dKs-1nQrPK-1}\nwmargintag{{\nwtagstyle{}\subpageref{NW1v8dKs-1nQrPK-1}}}\moddef{module definition~{\nwtagstyle{}\subpageref{NW1v8dKs-1nQrPK-1}}}\endmoddef\nwstartdeflinemarkup\nwusesondefline{\\{NW1v8dKs-To6Q8-1}}\nwprevnextdefs{\relax}{NW1v8dKs-1nQrPK-2}\nwenddeflinemarkup
//
// Bit Node Module definition
module mkBitNode (BitNode);
   \LA{}state bitNode~{\nwtagstyle{}\subpageref{NW1v8dKs-3Q7ADd-1}}\RA{}
   \LA{}rules bitNode~{\nwtagstyle{}\subpageref{NW1v8dKs-2ke5KA-1}}\RA{}
   \LA{}interfaces bitNode~{\nwtagstyle{}\subpageref{NW1v8dKs-469JCS-1}}\RA{}
\LA{}wrap up bitNode~{\nwtagstyle{}\subpageref{NW1v8dKs-oZLgz-1}}\RA{}
// -----------------------------------------------------------------

\nwalsodefined{\\{NW1v8dKs-1nQrPK-2}}\nwused{\\{NW1v8dKs-To6Q8-1}}\nwendcode{}\nwbegindocs{13}\nwdocspar
The input FIFO {\Tt{}ffCodeIn\nwendquote} receives the symbol of the code word meant for this
bit-node. The output FIFO {\Tt{}ffDataOut\nwendquote} holds the decoded symbol.

\nwenddocs{}\nwbegincode{14}\sublabel{NW1v8dKs-3Q7ADd-1}\nwmargintag{{\nwtagstyle{}\subpageref{NW1v8dKs-3Q7ADd-1}}}\moddef{state bitNode~{\nwtagstyle{}\subpageref{NW1v8dKs-3Q7ADd-1}}}\endmoddef\nwstartdeflinemarkup\nwusesondefline{\\{NW1v8dKs-1nQrPK-1}}\nwprevnextdefs{\relax}{NW1v8dKs-3Q7ADd-2}\nwenddeflinemarkup
// Sub-modules and state
// Input FIFO - code word
FIFO  #(Symbol)   ffCodeIn    <- mkFIFO;

// Output FIFO - decoded code word
FIFO  #(Symbol)   ffDataOut   <- mkFIFO;

\nwalsodefined{\\{NW1v8dKs-3Q7ADd-2}\\{NW1v8dKs-3Q7ADd-3}}\nwused{\\{NW1v8dKs-1nQrPK-1}}\nwendcode{}\nwbegindocs{15}\nwdocspar
The {\Tt{}ffB2C\nwendquote} FIFO holds the partially processed codeword. The contents of
this FIFO will be consumed by the check node when they are ready. The
{\Tt{}vffC2B\nwendquote} FIFO receives partially processed codewords from the check nodes
for the next iteration of processing.

\nwenddocs{}\nwbegincode{16}\sublabel{NW1v8dKs-3Q7ADd-2}\nwmargintag{{\nwtagstyle{}\subpageref{NW1v8dKs-3Q7ADd-2}}}\moddef{state bitNode~{\nwtagstyle{}\subpageref{NW1v8dKs-3Q7ADd-1}}}\plusendmoddef\nwstartdeflinemarkup\nwusesondefline{\\{NW1v8dKs-1nQrPK-1}}\nwprevnextdefs{NW1v8dKs-3Q7ADd-1}{NW1v8dKs-3Q7ADd-3}\nwenddeflinemarkup
// Partially processed codeword meant for the checknodes
FIFO  #(Symbol)      ffB2C       <- mkFIFO;

// Partially processed codeword from the checknodes
Vector #(
     NConnections
   , FIFO #(Symbol)) vffC2B      <- replicateM (mkFIFO);

\nwused{\\{NW1v8dKs-1nQrPK-1}}\nwendcode{}\nwbegindocs{17}\nwdocspar
Behaviour is described in terms of atomic sets of actions called rules.
The rule, {\Tt{}rlProcessFirstIteration\nwendquote} executes the actions for the first
iteration of processing a new code word. 

\begin{itemize}
\item Consume the codeword which is currently in {\Tt{}ffCodeIn\nwendquote}
\item Carry out some initial processing on the codeword
\item Enqueue the result into the {\Tt{}ffB2C\nwendquote} for checknode processing
\item Update the iteration count - this is updated by $2$ as the count is
maintained only in the bit node, and the check node acts as a purely
passive device.
\end{itemize}

The first iteration is counted when the code word goes through the check
node for the first time.

\nwenddocs{}\nwbegincode{18}\sublabel{NW1v8dKs-2ke5KA-1}\nwmargintag{{\nwtagstyle{}\subpageref{NW1v8dKs-2ke5KA-1}}}\moddef{rules bitNode~{\nwtagstyle{}\subpageref{NW1v8dKs-2ke5KA-1}}}\endmoddef\nwstartdeflinemarkup\nwusesondefline{\\{NW1v8dKs-1nQrPK-1}}\nwprevnextdefs{\relax}{NW1v8dKs-2ke5KA-2}\nwenddeflinemarkup
// Rules and behaviour
\LA{}functions bitNode~{\nwtagstyle{}\subpageref{NW1v8dKs-2TR1lR-1}}\RA{}

// Rule to process the first iteration of a new code word
rule rlProcessFirstIteration (rgIterationCount == 0);
   // As this is the first iteration, consume the codeword which is
   // currently in ffCodeIn. Carry out the computation on the codeword
   let codeIn = ffCodeIn.first; ffCodeIn.deq;
   let codeOut = fnInitialBitNodeProcessing (codeIn);

   // Send the output to the check nodes
   ffB2C.enq (codeOut);
   
   // Bookkeeping - keep track of iterations to know when to stop. The
   // first iteration is treated as the first time the code word goes
   // through the check node
   rgIterationCount  <= rgIterationCount + 1;
endrule

\nwalsodefined{\\{NW1v8dKs-2ke5KA-2}}\nwused{\\{NW1v8dKs-1nQrPK-1}}\nwendcode{}\nwbegindocs{19}\nwdocspar
The register, {\Tt{}rgIterationCount\nwendquote} keeps track of the number of iterations
for this particular code word. When a certain prefixed iteration limit is
reached, processing stops. This is a system-wide setting.

\nwenddocs{}\nwbegincode{20}\sublabel{NW1v8dKs-3Q7ADd-3}\nwmargintag{{\nwtagstyle{}\subpageref{NW1v8dKs-3Q7ADd-3}}}\moddef{state bitNode~{\nwtagstyle{}\subpageref{NW1v8dKs-3Q7ADd-1}}}\plusendmoddef\nwstartdeflinemarkup\nwusesondefline{\\{NW1v8dKs-1nQrPK-1}}\nwprevnextdefs{NW1v8dKs-3Q7ADd-2}{\relax}\nwenddeflinemarkup
Reg   #(Bit #(4))                rgIterationCount  <- mkReg (0);

// -----------------------------------------------------------------

\nwused{\\{NW1v8dKs-1nQrPK-1}}\nwendcode{}\nwbegindocs{21}\nwdocspar
The function {\Tt{}fnBitNodeProcessing\nwendquote}, carries out the actual bit
manipulation of the codeword symbols as per the min-sum-algorithm.

\nwenddocs{}\nwbegincode{22}\sublabel{NW1v8dKs-2TR1lR-1}\nwmargintag{{\nwtagstyle{}\subpageref{NW1v8dKs-2TR1lR-1}}}\moddef{functions bitNode~{\nwtagstyle{}\subpageref{NW1v8dKs-2TR1lR-1}}}\endmoddef\nwstartdeflinemarkup\nwusesondefline{\\{NW1v8dKs-2ke5KA-1}}\nwprevnextdefs{\relax}{NW1v8dKs-2TR1lR-2}\nwenddeflinemarkup
function Symbol fnBitNodeProcessing (Vector #(NConnections, Symbol) x);
   // XXX This is at present a dummy function which simply does an
   // XOR of the symbols
   // It will actually do some sort of a zip function which goes:
   // Vector#(N,Symbol) -> Symbol
   return (foldl1 (\\^ , x));
endfunction

\nwalsodefined{\\{NW1v8dKs-2TR1lR-2}}\nwused{\\{NW1v8dKs-2ke5KA-1}}\nwendcode{}\nwbegindocs{23}\nwdocspar
The function {\Tt{}fnInitialBitNodeProcessing\nwendquote}, does not work on a vector
of Symbols like {\Tt{}fnBitNodeProcessing\nwendquote}. It can be thought of as a
function which prepares the input symbol for iterative processing. 

\nwenddocs{}\nwbegincode{24}\sublabel{NW1v8dKs-2TR1lR-2}\nwmargintag{{\nwtagstyle{}\subpageref{NW1v8dKs-2TR1lR-2}}}\moddef{functions bitNode~{\nwtagstyle{}\subpageref{NW1v8dKs-2TR1lR-1}}}\plusendmoddef\nwstartdeflinemarkup\nwusesondefline{\\{NW1v8dKs-2ke5KA-1}}\nwprevnextdefs{NW1v8dKs-2TR1lR-1}{\relax}\nwenddeflinemarkup

function Symbol fnInitialBitNodeProcessing (Symbol x);
   // XXX This is at present a dummy function which simply does an
   // forwards the signal to the output
   return (x);
endfunction

\nwused{\\{NW1v8dKs-2ke5KA-1}}\nwendcode{}\nwbegindocs{25}\nwdocspar
The rule {\Tt{}rlProcessRemIteration\nwendquote} executes the actions for the remaining
iterations for processing a code word. The input for these iterations is
from the partially processed word in {\Tt{}vffC2B\nwendquote}.

\begin{itemize}
\item Consume the codeword which is currently in {\Tt{}vffC2B\nwendquote}
\item Carry out the computation on the codeword
\item Enqueue the result into the {\Tt{}ffB2C\nwendquote} for checknode processing
\item Update the iteration count - this is updated by $2$ as the count is
maintained only in the bit node, and the check node acts as a purely
passive node.
\end{itemize}

\nwenddocs{}\nwbegincode{26}\sublabel{NW1v8dKs-2ke5KA-2}\nwmargintag{{\nwtagstyle{}\subpageref{NW1v8dKs-2ke5KA-2}}}\moddef{rules bitNode~{\nwtagstyle{}\subpageref{NW1v8dKs-2ke5KA-1}}}\plusendmoddef\nwstartdeflinemarkup\nwusesondefline{\\{NW1v8dKs-1nQrPK-1}}\nwprevnextdefs{NW1v8dKs-2ke5KA-1}{\relax}\nwenddeflinemarkup
// Rule to process remaining iterations
rule rlProcessRemIteration (
      (rgIterationCount > 0)
   && (rgIterationCount < fromInteger (valueOf (NIterations))));
   // As this iteration works of a partial result from the checknode,
   // the input comes from the vector of fifos vffC2B
   Vector #(NConnections, Symbol) codeIn;
   for (Integer i=0; i<valueOf(NConnections); i=i+1) begin
      codeIn[i] = vffC2B[i].first;
      vffC2B[i].deq;
   end

   let codeOut = fnBitNodeProcessing (codeIn);

   // Bookkeeping - keep track of iterations to know when to stop
   if (rgIterationCount == (fromInteger (valueOf (NIterations))-1)) begin
      // time to stop
      rgIterationCount <= 0;

      // Send the processed code word to the output
      ffDataOut.enq (codeOut);
   end
   
   // continue with more iterations
   else begin
      rgIterationCount  <= rgIterationCount + 2;

      // Send the output to the check nodes
      ffB2C.enq (codeOut);
   end
endrule

// -----------------------------------------------------------------

\nwused{\\{NW1v8dKs-1nQrPK-1}}\nwendcode{}\nwbegindocs{27}\nwdocspar
The {\Tt{}rlProcessRemIteration\nwendquote} rule also needs to check if the iterations
are complete. If so, it should reset the {\Tt{}rgIterationCount\nwendquote} and
instead of sending the output to the {\Tt{}ffB2C\nwendquote} it should sent it to
{\Tt{}ffDataOut\nwendquote}.


\nwenddocs{}\nwbegindocs{28}\nwdocspar
Creating the interfaces simply involves stitching up the connections to the
input and output FIFOs using library functions -- {\Tt{}toPut\nwendquote}, {\Tt{}toGet\nwendquote}.
Since the {\Tt{}c2b\nwendquote} interface is a vector, the {\Tt{}map\nwendquote} higher-order function is
applied.

\nwenddocs{}\nwbegincode{29}\sublabel{NW1v8dKs-469JCS-1}\nwmargintag{{\nwtagstyle{}\subpageref{NW1v8dKs-469JCS-1}}}\moddef{interfaces bitNode~{\nwtagstyle{}\subpageref{NW1v8dKs-469JCS-1}}}\endmoddef\nwstartdeflinemarkup\nwusesondefline{\\{NW1v8dKs-1nQrPK-1}}\nwenddeflinemarkup
// Interface
interface codeIn     = toPut (ffCodeIn);
interface dataOut    = toGet (ffDataOut);
interface c2b        = map (toPut, vffC2B);
interface b2c        = toGet (ffB2C);

// -----------------------------------------------------------------

\nwused{\\{NW1v8dKs-1nQrPK-1}}\nwendcode{}\nwbegincode{30}\sublabel{NW1v8dKs-oZLgz-1}\nwmargintag{{\nwtagstyle{}\subpageref{NW1v8dKs-oZLgz-1}}}\moddef{wrap up bitNode~{\nwtagstyle{}\subpageref{NW1v8dKs-oZLgz-1}}}\endmoddef\nwstartdeflinemarkup\nwusesondefline{\\{NW1v8dKs-1nQrPK-1}}\nwenddeflinemarkup
endmodule : mkBitNode

\nwused{\\{NW1v8dKs-1nQrPK-1}}\nwendcode{}\nwbegindocs{31}\nwdocspar
\section{The Check Node}
The {\Tt{}mkCheckNode\nwendquote} module receives the partially decoded code word from
the {\Tt{}mkBitNode\nwendquote}. It operates in \emph{slave} mode and processes all
inputs in the same manner. The {\Tt{}mkCheckNode\nwendquote} does not keep track of
iterations. The {\Tt{}mkCheckNode\nwendquote} provides an interface of type {\Tt{}CheckNode\nwendquote}.

\nwenddocs{}\nwbegincode{32}\sublabel{NW1v8dKs-1nQrPK-2}\nwmargintag{{\nwtagstyle{}\subpageref{NW1v8dKs-1nQrPK-2}}}\moddef{module definition~{\nwtagstyle{}\subpageref{NW1v8dKs-1nQrPK-1}}}\plusendmoddef\nwstartdeflinemarkup\nwusesondefline{\\{NW1v8dKs-To6Q8-1}}\nwprevnextdefs{NW1v8dKs-1nQrPK-1}{\relax}\nwenddeflinemarkup
//
// Check Node Module definition
module mkCheckNode (CheckNode);
   \LA{}state checkNode~{\nwtagstyle{}\subpageref{NW1v8dKs-2jCWTO-1}}\RA{}
   \LA{}rules checkNode~{\nwtagstyle{}\subpageref{NW1v8dKs-4bVYKS-1}}\RA{}
   \LA{}interfaces checkNode~{\nwtagstyle{}\subpageref{NW1v8dKs-1mJizB-1}}\RA{}
   \LA{}wrap up checkNode~{\nwtagstyle{}\subpageref{NW1v8dKs-HAQpJ-1}}\RA{}
// -----------------------------------------------------------------

\nwused{\\{NW1v8dKs-To6Q8-1}}\nwendcode{}\nwbegindocs{33}\nwdocspar
The input FIFOs {\Tt{}vffB2C\nwendquote} receives the partially processed part of the code
word meant for this check-node. The output FIFO {\Tt{}ffC2B\nwendquote} holds the
partially decoded code word.

\nwenddocs{}\nwbegincode{34}\sublabel{NW1v8dKs-2jCWTO-1}\nwmargintag{{\nwtagstyle{}\subpageref{NW1v8dKs-2jCWTO-1}}}\moddef{state checkNode~{\nwtagstyle{}\subpageref{NW1v8dKs-2jCWTO-1}}}\endmoddef\nwstartdeflinemarkup\nwusesondefline{\\{NW1v8dKs-1nQrPK-2}}\nwenddeflinemarkup
// Sub-modules and state
// Input FIFO - code word
Vector #(
     NConnections
   , FIFO #(Symbol)) vffB2C        <- replicateM (mkFIFO);

// Output FIFO - decoded code word
FIFO  #(Symbol)      ffC2B         <- mkFIFO;

\nwused{\\{NW1v8dKs-1nQrPK-2}}\nwendcode{}\nwbegindocs{35}\nwdocspar
The rule {\Tt{}rlProcessIteration\nwendquote} executes the actions for process the
input from the bit nodes.

\begin{itemize}
\item Consume the codeword which is currently in {\Tt{}vffB2C\nwendquote}
\item Carry out the computation on the codeword
\item Enqueue the result into the {\Tt{}ffC2B\nwendquote} for bit-node processing
\end{itemize}

\nwenddocs{}\nwbegincode{36}\sublabel{NW1v8dKs-4bVYKS-1}\nwmargintag{{\nwtagstyle{}\subpageref{NW1v8dKs-4bVYKS-1}}}\moddef{rules checkNode~{\nwtagstyle{}\subpageref{NW1v8dKs-4bVYKS-1}}}\endmoddef\nwstartdeflinemarkup\nwusesondefline{\\{NW1v8dKs-1nQrPK-2}}\nwenddeflinemarkup
// Rules and behaviour
\LA{}functions checkNode~{\nwtagstyle{}\subpageref{NW1v8dKs-oud5A-1}}\RA{}
rule rlProcessIteration;
   // get the partial result
   Vector #(NConnections, Symbol) codeIn;
   for (Integer i=0; i<valueOf(NConnections); i=i+1) begin
      codeIn[i] = vffB2C[i].first;
      vffB2C[i].deq;
   end

   // Process the partial result further
   let codeOut = fnCheckNodeProcessing (codeIn);

   // Send the partial result to the bit node
   ffC2B.enq (codeOut);
endrule

// -----------------------------------------------------------------

\nwused{\\{NW1v8dKs-1nQrPK-2}}\nwendcode{}\nwbegindocs{37}\nwdocspar
The function {\Tt{}fnCheckNodeProcessing\nwendquote}, carries out the actual bit
manipulation of the codeword bits as per the min-sum algorithm.

\nwenddocs{}\nwbegincode{38}\sublabel{NW1v8dKs-oud5A-1}\nwmargintag{{\nwtagstyle{}\subpageref{NW1v8dKs-oud5A-1}}}\moddef{functions checkNode~{\nwtagstyle{}\subpageref{NW1v8dKs-oud5A-1}}}\endmoddef\nwstartdeflinemarkup\nwusesondefline{\\{NW1v8dKs-4bVYKS-1}}\nwenddeflinemarkup
function Symbol fnCheckNodeProcessing (Vector #(NConnections, Symbol) x);
   // XXX This is at present a dummy function which simply does an
   // inverting of the bits
   // It will actually do some sort of a zip function which goes:
   // Vector#(N,Symbol) -> Symbol
   return (foldl1 (\\^ , x));
endfunction

\nwused{\\{NW1v8dKs-4bVYKS-1}}\nwendcode{}\nwbegindocs{39}\nwdocspar
Creating the interfaces simply involves stitching up the connections to the
input and output FIFOs using library functions -- {\Tt{}toPut\nwendquote} and {\Tt{}toGet\nwendquote}.
Since the {\Tt{}b2c\nwendquote} interface is a vector, the {\Tt{}map\nwendquote} higher-order function is
applied.

\nwenddocs{}\nwbegincode{40}\sublabel{NW1v8dKs-1mJizB-1}\nwmargintag{{\nwtagstyle{}\subpageref{NW1v8dKs-1mJizB-1}}}\moddef{interfaces checkNode~{\nwtagstyle{}\subpageref{NW1v8dKs-1mJizB-1}}}\endmoddef\nwstartdeflinemarkup\nwusesondefline{\\{NW1v8dKs-1nQrPK-2}}\nwenddeflinemarkup
// Interface
interface c2b = toGet (ffC2B);
interface b2c = map (toPut, vffB2C);

// -----------------------------------------------------------------

\nwused{\\{NW1v8dKs-1nQrPK-2}}\nwendcode{}

\nwixlogsorted{c}{{boilerplate}{NW1v8dKs-To6Q8-1}{\nwixd{NW1v8dKs-To6Q8-1}}}%
\nwixlogsorted{c}{{functions bitNode}{NW1v8dKs-2TR1lR-1}{\nwixu{NW1v8dKs-2ke5KA-1}\nwixd{NW1v8dKs-2TR1lR-1}\nwixd{NW1v8dKs-2TR1lR-2}}}%
\nwixlogsorted{c}{{functions checkNode}{NW1v8dKs-oud5A-1}{\nwixu{NW1v8dKs-4bVYKS-1}\nwixd{NW1v8dKs-oud5A-1}}}%
\nwixlogsorted{c}{{interface definition}{NW1v8dKs-3l1BOL-1}{\nwixu{NW1v8dKs-4O0RoX-1}\nwixd{NW1v8dKs-3l1BOL-1}\nwixd{NW1v8dKs-3l1BOL-2}\nwixd{NW1v8dKs-3l1BOL-3}}}%
\nwixlogsorted{c}{{interfaces bitNode}{NW1v8dKs-469JCS-1}{\nwixu{NW1v8dKs-1nQrPK-1}\nwixd{NW1v8dKs-469JCS-1}}}%
\nwixlogsorted{c}{{interfaces checkNode}{NW1v8dKs-1mJizB-1}{\nwixu{NW1v8dKs-1nQrPK-2}\nwixd{NW1v8dKs-1mJizB-1}}}%
\nwixlogsorted{c}{{module definition}{NW1v8dKs-1nQrPK-1}{\nwixu{NW1v8dKs-To6Q8-1}\nwixd{NW1v8dKs-1nQrPK-1}\nwixd{NW1v8dKs-1nQrPK-2}}}%
\nwixlogsorted{c}{{rules bitNode}{NW1v8dKs-2ke5KA-1}{\nwixu{NW1v8dKs-1nQrPK-1}\nwixd{NW1v8dKs-2ke5KA-1}\nwixd{NW1v8dKs-2ke5KA-2}}}%
\nwixlogsorted{c}{{rules checkNode}{NW1v8dKs-4bVYKS-1}{\nwixu{NW1v8dKs-1nQrPK-2}\nwixd{NW1v8dKs-4bVYKS-1}}}%
\nwixlogsorted{c}{{state bitNode}{NW1v8dKs-3Q7ADd-1}{\nwixu{NW1v8dKs-1nQrPK-1}\nwixd{NW1v8dKs-3Q7ADd-1}\nwixd{NW1v8dKs-3Q7ADd-2}\nwixd{NW1v8dKs-3Q7ADd-3}}}%
\nwixlogsorted{c}{{state checkNode}{NW1v8dKs-2jCWTO-1}{\nwixu{NW1v8dKs-1nQrPK-2}\nwixd{NW1v8dKs-2jCWTO-1}}}%
\nwixlogsorted{c}{{type definitions}{NW1v8dKs-4O0RoX-1}{\nwixu{NW1v8dKs-To6Q8-1}\nwixd{NW1v8dKs-4O0RoX-1}}}%
\nwixlogsorted{c}{{wrap up bitNode}{NW1v8dKs-oZLgz-1}{\nwixu{NW1v8dKs-1nQrPK-1}\nwixd{NW1v8dKs-oZLgz-1}}}%
\nwixlogsorted{c}{{wrap up checkNode}{NW1v8dKs-HAQpJ-1}{\nwixu{NW1v8dKs-1nQrPK-2}\nwixd{NW1v8dKs-HAQpJ-1}}}%
\nwbegincode{41}\sublabel{NW1v8dKs-HAQpJ-1}\nwmargintag{{\nwtagstyle{}\subpageref{NW1v8dKs-HAQpJ-1}}}\moddef{wrap up checkNode~{\nwtagstyle{}\subpageref{NW1v8dKs-HAQpJ-1}}}\endmoddef\nwstartdeflinemarkup\nwusesondefline{\\{NW1v8dKs-1nQrPK-2}}\nwenddeflinemarkup
endmodule : mkCheckNode
endpackage
\nwused{\\{NW1v8dKs-1nQrPK-2}}\nwendcode{}

\end{document}
